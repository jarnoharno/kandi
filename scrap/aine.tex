\documentclass[a4paper,parskip=half]{scrartcl}
\usepackage[finnish]{babel}
\usepackage[utf8]{inputenc}
\usepackage{verbatim}
\usepackage[numbers]{natbib}
\usepackage{hyperref}
\usepackage[nonumberlist,acronym]{glossaries}

\makeglossaries
\newglossaryentry{laskelmasuunnistus}{
  name={laskelmasuunnistus},
  description={dead reckoning}
}
\newglossaryentry{partikkelisuodin}{
  name={partikkelisuodin},
  description={particle filter}
}


\subject{Aine}
\author{Jarno Leppänen}
\title{Sisätilapaikannus matkapuhelimella}
\date{14.3.2014}

\begin{document}

\maketitle

%\tableofcontents

\section{Johdanto}

Mobiililaitteiden yleistymisen myötä kiinnostus paikkatietoa hyödyntäviä
sovelluksia kohtaan on kasvanut\cite{harle2013survey}. Tyypillinen älypuhelin
tarjoaa tänä päivänä käyttäjälleen esimerkiksi paikallisen sääennusteen, ohjaa
autoilijan perille välttäen pahimpia ruuhkia sekä seuraa kuntoilijan
kalorikulutusta ja juoksureittiä kartalla.

Tarkkaan paikkatietoon perustuvien sovellusten yleistymisen on mahdollistanut
satelliittipaikannuksen (\acrshort{GNSS}) ja erityisesti
\acrshort{GPS}-järjestelmän käyttöönotto mobiililaitteissa. Ulkotiloissa
satelliittipaikannuksella saavutetaan erinomainen tarkkuus ja globaali
kattavuus, mutta sisätiloissa satelliittipaikannus toimii huonosti, sillä se
vaatii lähes esteettömän näkymän vastaaottimen ja satelliitin välillä. Suurin
osa mobiililaitteiden käyttäjien ajasta kuluu kuitenkin sisätiloissa, joten
sisätilapaikannusmenetelmien tutkimukseen on panostettu viime vuosina
merkittävästi.

Sisätilapaikannukselle on löydettävissä useita
sovellusmahdollisuuksia\cite{mautz2012indoor}:
\begin{itemize}
  \item Paikkatietoon perustuvat palvelut (\acrshort{LBS}), jossa sovellus
    hyödyntää paikkatietoa tarjoten käyttäjälle kontekstiriippuvaista
    informaatiota. Tällaisia palveluita voivat olla esimerkiksi
    navigointipalvelut kauppakeskuksissa ja messuilla,
    käyttäjää mahdollisesti kiinnostavien tuotteiden ja tarjousten esittely
    kaupan tietyllä hyllyosastolla tai pysäkkikohtaisen aikataulun näyttäminen
    bussipysäkillä.
  \item Sosiaalisen median palvelut, kuten kumppanin löytäminen tai
    kommunikointi ja toimintojen koordinointi ystävien kanssa.
  \item Resurssien käytön tehostaminen optimoimalla rakennuste valaistusta ja
    lämmitystä käyttäjien sijainnin perusteella.
  \item Turvallisuuden parantaminen: käyttäjien sijainnin löytäminen
    ja ohjaus poistumisteille hätätilanteessa ja palokunnan navigointi
    sammutustehtävissä.
\end{itemize}

Sisätilapaikannukseen on kehitetty lukuisia menetelmiä ja alaan liittyvää
kirjallisuutta on runsaasti. Menetelmät eroavat toisistaan huomattavasti
kattavuuden, tarkkuuden, infrastruktuurivaatimusten ja järjestelmän hinnan
osalta, eikä mikään niistä ole satelliittipaikannuksen tavoin saavuttanut
merkittävää jalansijaa.

Nykyisisten mobiililaitteiden ominaisuuksien ja olemassa olevan infrastruktuurin
kannalta oleelliset sisätilapaikannukseen kehitetyt menetelmät perustuvat
tietoliikenneyhteyksiin käytettävien langattomien verkkojen avulla tehtävään
paikannukseen sekä mobiililaittessa olevien
kiihtyvyysantureiden, gyroskoopin ja magneettikompassin avulla tehtävään
inertiaalinavigaatioon.

Langattomia verkkoja hyödyntävät paikannusmenetelmät perustuvat GSM-, WiFi-
ja Bluetooth- verkkojen havainnointiin. Tietoliikenneverkkojen suunnittelussa 
pyritään kuitenkin maksimoimaan verkon kattavuus
ja minimoimaan tukiasemien kuuluvuusalueiden päällekkäisyys, mikä on
vaikeuttanut näiden menetelmien käyttöönottoa. Hiljattain markkinoille tulleet
\acrshort{BLE}-tekniikkaan perustuvat edulliset Bluetooth-majakat, joita
asennetaan eritoten \acrshort{LBS}-palveluita silmällä pitäen saattavat
tulevaisuudessa helpottaa merkittävästi sisätilapaikannusta
\cite{sand2014positioning}.

Mobiililaitteiden sensoreilla tehtävä inertiaalinavigaatio on minimaalisten
infrastruktuurivaatimustensa takia kiinnostava teknologia sisätiloissa
tapahtuvaan paikannukseen. Näiden menetelmien ongelmana on mobiililaitteissa
käytettyjen \acrshort{MEMS}-sen\-so\-rei\-den e\-pä\-tark\-kuus ja toisaalta menetelmälle
ominainen paikannusvirheen kumuloituminen. Inertiaalinavigaatio tarvitseekin
tyypillisesti jonkin absoluuttisen paikannusmenetelmän avukseen, jotta
paikannuksen virhe ei kasva liian suureksi. Inertiaalinavigaatiota on
viime aikoina hyödynnetty lupaavasti myös langattomiin verkkoihin perustuvassa
paikannuksessa tarvittavaan radiosignaalikartan muodostamiseen.

\printglossary[type=\acronymtype,title=Lyhenteet]

\bibliographystyle{plainnat}
\bibliography{kandi}
\end{document}
