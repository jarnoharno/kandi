\documentclass[a4paper]{scrartcl}
\usepackage[finnish]{babel}
\usepackage[utf8]{inputenc}
\usepackage{verbatim}
\usepackage[numbers]{natbib}
\usepackage[xindy,nonumberlist,acronym]{glossaries}
\usepackage{hyperref}
\subject{Aine}
\author{Jarno Leppänen}
\title{Sisätilapaikannus matkapuhelimella}
\date{18.2.2014}
\begin{document}

\maketitle
\makeglossaries

\newglossaryentry{laskelmasuunnistus}{
  name={laskelmasuunnistus},
  description={dead reckoning}
}
\newglossaryentry{partikkelisuodin}{
  name={partikkelisuodin},
  description={particle filter}
}


\tableofcontents

\section{Johdanto}

\section{Absoluuttiset paikannustekniikat}
\subsection{GPS}
\cite{groves2013principles}
\subsection{Radiolähteet}
\cite{varshavsky2005gsm}
\subsubsection{Kalibrointi}
\begin{itemize}
\item HORUS
\item EZ
\end{itemize}

\section{Suhteelliset paikannustekniikat}
\subsection{Inertiaalinavigointi}
\begin{itemize}
\item Kumuloituva virhe
\item Virheenkorjaus rajoitusehdoilla
\end{itemize}
\cite{groves2013principles}
\subsection{Jalankulkijan laskelmasuunnistus}
\cite{harle2013survey}
\subsubsection{SHS}
\begin{itemize}
\item Askeleen tunnistus
\item Askeleen pituuden estimointi
\item Suunnan estimointi
\item ZUPT-rajoitusehto
\end{itemize}

\section{Hybriditekniikat}
\cite{woodman2010pedestrian}
\cite{evennou2006advanced}
\cite{leppakoski2013pedestrian}
\subsection{Bayes-suotimet}
\subsubsection{Kalman-suodin}
\subsubsection{Partikkelisuotimet}
\subsection{Karttatieto rajoitusehtona}
\cite{li2012reliable}
\subsection{Radiotunnistetiedon ja laskelmasuunnistuksen yhdistäminen}
\cite{rai2012zee}

\section{Matkapuhelinsovellukset}
\subsection{Absoluuttisen paikannukseen perustuvat navigaattorit}
\subsubsection{Kokeiluasteella olevat järjestelmät}
Bluetooth, RFID, magneettikenttä\ldots
\subsection{Askelmittarit}
Runtastic, jne.
\subsection{Kokeiluasteella oleva hybridijärjestelmät}
Zee?

\section{Yhteenveto}

\section{Referaatit}
asdfasdafsdasasdf

\subsection{Varshavsky \cite{varshavsky2005gsm}}
\begin{itemize}
  \item Kolme erilaista käyttötarkoitusta: sisätilapaikannus, ulkotilapaikannus
    sekä paikan tunnistus.
  \item Väite: matkapuhelinpaikannus toimii hyvin sisätilapaikannukseen ja
    paikan tunnistukseen.
\end{itemize}

\subsection{Harle \cite{harle2013survey}}
\begin{itemize}
  \item Väite: PDR:stä apua sisätilapaikannukseen, tarvitaan ajoittaisia fiksejä
    absoluuttisesta paikannusjärjestelmästä.
  \item Sisätilapaikannuksen sovelluksia:
    \begin{itemize}
      \item Turvallisuus, pelastautuminen
      \item Tietoturva, paikkaan perustuva resurssien hallinta
      \item Tehokkuus: valaistuksen, lämmityksen optimointi
      \item Automaattinen resurssien ohjaus: esim. puheluiden ohjaus
        lähimpään laitteeseen.
      \item Navigointi: vieraan ohjaus
    \end{itemize}
  \item Väite: olemassa oleviin radiosignaaleihin perustuva paikannus ei
    riitä: tukiasemien sijoittelua ei ajateltu paikannuksen kannalta.
  \item PDR nousussa, koska sensorit yleistyneet.
  \item Sisätilapaikannustyyppejä:
     \begin{itemize}
       \item Suuntima- ja etäisyysjärjestelmät (lateration and angulation):
         ultraääni, laajakaistaradio, Ubisense
       \item Läheisyysantureihin perustuvat järjestelmät: RFID, bluetooth
         (erityisen suosittu valmistajien keskuudessa),
         In-Location Alliance, iBeacon,
       \item Radiotunnistejärjestelmät: Wifi fingerprinting (Harlen mukaan
         tähän mennessä menestyksekkäin). Ongelmana ristiriita paikannuksen
         ja tietoliikenteen tarpeissa: paikannusta varten referenssipisteitä
         tulisi olla useita, mutta tietoliikenteen kannalta halutaan
         mahdollisimman vähän päällekkäistä peittoa.
       \item Laskelmasuunnistus: ei infraa, tarvitsee tuekseen absoluuttista
         paikannusjärjestelmää suhteellinen sijainnin vuoksi,
         virheen kumuloituminen ongelma
     \end{itemize}
   \item Inertiaalinavigointi (INS), joka laskee sensorin kulkemaa trajektoria
     vs. askelsuuntajärjestelmä (SHS, Step-and-Heading), joka laskee
     käyttäjän mukana olevan sensorin avulla askelten lukumärää, pituutta ja
     suuntaa.
   \item SHS-INS: INS, joka pilkotaan askeleiksi.
   \item Robotiikan paikannusmenetelmiä:
     \begin{itemize}
       \item Partikkelisuotimet: eliminoidaan rajoitusehtojen estämiä
         tiloja, karttatieto rajoitusehtona.
       \item Yhtäaikainen paikannus ja kartoitus (SLAM): paikannustieto paranee
         ajan myötä.
     \end{itemize}
   \item Askelluksen analysointi: stance-swing, toe-off/push-off,
     heel-strike/foot-down, jaksollinen liike
   \item Askeleen havainnointi: stance detection, peak detection, zero crossing
     autocorrelation, spectral analysis
   \item INS PDR-kontekstissa: otettava gyroskooppidata huomioon, kolmas
     integrointi, MEMS-sensoreilla virhe dominoi minuutin tai kahden
     kuluttua. Klassisesti käytetään 15 muuttujan laajennettua Kalman-suodinta:
     3 tilaa * kiihtyvyys, nopeus, suunta, kiihtyvyysanturi, gyroskooppi.
   \item INS-rajoitusehto: ZUPT, tunistetaan, milloin jalka on paikallaan.
     Sensorin täytyy tällöin olla kiinnitettynä jalkaterään. NavShoe, 2005.
   \item Magnetometrit: absoluuttinen suunta. Ongelmana heikko magneettikenttä
     ja rakennusten sisällä tavattavat magneettiset häiriöt. Yleinen tapa
     virheenkorjaukseen on suodattaa toleranssin sisällä olevat "mahdottomat"
     arvot. Yhdistämällä gyroskooppi- ja magnetometridataa voidaan korjata
     magnetometrin virhettä.
   \item Hybridijärjestelmät: \textbf{PDR + Fingerprinting -tekniikalla voidaan
     kartoittaa tilaa ilman erillistä kalibrointivaihetta.} (Zee, Harle/Woodman)
     WifiSlam, Faragher Opportunistic radio SLAM
   \item Kysymyksiä: vertailukelpoinen evaluointimenetelmä, miten huomioida
     sensorin vapaa sijoittelu, sensorin dynaaminen virhe, virrankulutus,
     laskentaresurssien kulutus, järjestelmän initialisointi (bootstrapping):
     karttatieto, partikkelit
   \item Hybridijärjestelmien esiinmarssi: PDR + jokin absoluuttinen paikannus
     (Wifi, bluetootk, GSM, RFID).
\end{itemize}

\subsection{Woodman PhD\cite{woodman2010pedestrian}}
\begin{itemize}
  \item Esitys absoluuttisista ja suhteellisista paikannusmenetelmistä
  \item Kehitetty menetelmä: jalkapohjaan kiinnitettävä anturi, PDR-suodin
  \item Väite: kompassin käyttö suunnan estimoimiseksi mahdotonta
    magneettisten häiriöiden vuoksi.
  \item Karttatiedon hyödyntäminen partikkelisuotimen avulla. Ongelmana
    karttatiedon symmetriat, partikkelilukumäärän skaalautuvuus.
  \item Partikkelisuotimen avulla paikannusongelma palautuu luontevasti
    jäljitysongelmaksi partikkelipilven kasautuessa.
  \item Partikkelisuodin vaatii valtavasti laskentatehoa varsinkin epöavrmuuden
    ollessa suuri paikannuksen alkuvaiheessa.
  \item Partikkelien lukumäärän dynaaminen resamplaus, klusterin tunnistus,
    suotimen laskentatehokkuus. Paikannus on sitä tarkempi, mitä enemmän
    ympäristön rajoitusehtoja. Virheelliset rajoitusehdot ongelmana.
  \item Voronoi-verkko rajoitusehtona (käyttäjän sijainti voi olla ainoastaan
    tällä verkolla) voi dramaattisesti pienentää vaadittavien
    partikkelien lukumäärää.
  \item Avustettu paikannus: alustus, paikannuksen aikainen avustus. Suunta
    magnetometrin avulla, alkupaikka GPS:n tai muun absoluuttisen
    paikannusjärjestelmän avulla
  \item Wifi-fingerprinttien kartoittaminen automaattisesti PDR:n avulla:
    jaetaan kartta alueisiin, joihin rf-tunniste liitetään. Laajennettavissa
    muihin radiolähteisiin, joista saatavilla RSSI ja tunnistetiedot ja
    jotka pysyvät suhteellisen staattisina. Myös magneettisia häiriöitä
    voidaan kartoittaa.
  \item Väite: rf-signaalin havainnointi jäljitysvaiheessa hyödytöntä johtuen
    signaalikartan karkeasta resoluutiosta ja homogeenisuudesta.
  \item Johtopäätöksiä: INS mahdotonta - virhe 170m minuutin jälkeen. Suunta:
    partikkelisuodin ympäristön rajoitusehdoista, ei magnetometrista.
    Alustus ja partikkelien lukumäärän minimointi: absoluuttinen
    paikannusmenetelmä tai rf-tunnisteet.
  \item Ongelmia: Liikkuvat lattiatasot (hissit)?, kartoitus, sensorin
    sijaintiriippumattomuus, laskennalliset ja virankäytön vaatimukset.
\end{itemize}

\subsection{Rai\cite{rai2012zee}}
\begin{itemize}
  \item RF-kartoitus ongelma. Kehitetty radiosignaalin etenemiseen kehitettyjä
    malleja, jotka vähentävät kalibroinnin tarvetta. Ongelmia näissä ovat
    osittainen opetusdatan tarve, vaatimus tukiasemien konfiguraatio- ja
    tiedoista etukäteen sekä tarkkuuden menetys.
  \item Zee käyttää PDR:ää ja samanaikaista wifi-mittausta yhdessä karttatiedon
    kanssa radiotunnistekartan rakentamiseen.
  \item Novelties: (a) askeltunnistus ja suunnan estimointi (b) askellustiedon
    sisällyttäminen partikkelifiltteriin (c) menneisyyden paikkatietojen
    päivitys (d) Wifi-tietoon perustuva paikkatiedon alustus.
  \item Zee input: karttatieto (rajoitusehdot), käyttäjän trajektori
    PDR-menetelmillä.
  \item Partikkelifiltterin konvergenssia parannetaan wifi-fingerprint
    alustuksella sekä absoluuttisen suunnan estimoinnilla.
  \item Sijaintiriippumaton liikeentunnistus (PIME). Suuntapoikkeama (HO).
  \item Laajennettu partikkelisuodin, todennäköisyysjakauma sijainnin,
    askelpituuden ja suuntapoikkeaman suhteen.
  \item Zeessä kerättiin profiileja kiihtyvyysanturidatasta. Kävely tunnistetaan
    kiihtyvyysanturidatan autokorrelaatiosta sekä varianssista.
  \item Käyttäjän suunta tunnistetaan kiihtyvyysanturidatan yläsävelsarjan
    avulla: liikkeen suunnassa perustaajuus on voimakkaimmillaan ja
    liikettä vastaan kohtisuorassa suunnassa toinen ensimmäinen
    yläsäveltaajuus on voimakkaimmillaan.
\end{itemize}

\subsection{Li\cite{li2012reliable}}
\begin{itemize}
  \item Askeltunnistin: alipäästösuodin, peak detection + heuristiikkaa.
  \item Zeen kaltainen systeemi
\end{itemize}

\subsection{Evennou\cite{evennou2006advanced}}
\begin{itemize}
  \item Kalman-suodin wifi-paikannuksella ja PDR:llä.
\end{itemize}

\subsection{Groves\cite{groves2013principles}}
\begin{itemize}
  \item Kirja paikannuksesta
\end{itemize}

\subsection{Leppäkoski\cite{leppakoski2013pedestrian}}
\begin{itemize}
  \item CEKF PDR + WLAN, partikkelisuodin karttatiedolle.
  \item Vyöhön kiinnitetty sensori
\end{itemize}

\subsection{Woodman\cite{woodman2009rf}}
\begin{itemize}
  \item Partikkelisuotimen initialisointi RF-signaalin perusteella. Osa
    väitöskirjassa esitettyä menetelmää.
\end{itemize}

\subsection{Kunze\cite{kunze2009way}}
\begin{itemize}
  \item Käyttäjän kulkusuunnan estimointi kiihtyvyysanturidatasta
    pääkomponenttianalyysilla.
\end{itemize}

\subsection{Stirling\cite{stirling2003innovative}}
\begin{itemize}
  \item Kokoelma PDR-tekniikoita
  \item Kompassin kalibrointi
 \end{itemize}

\subsection{Steinhoff\cite{steinhoff2010dead}}
\begin{itemize}
  \item PDR taskussa olevalla sensorilla
  \item Askeltunnistus painovoiman suunnassa olevan kiihtyvyysanturidatan
    varianssista.
  \item Käyttäjän kulkusuunta gyroskooppidatasta (reiden pyörähtely) ja
    PCA.
\end{itemize}

\subsection{Wang\cite{wang2012unsupervised}}
\begin{itemize}
  \item UnLoc yhdistää laskelmasuunnistuksen, urban sensingin ja WiFi-osituksen.
  \item Puhelin kerää sensoridataa, josta klusteroidaan maamerkkejä.
  \item PDR: alipäästösuodin, etsitään vuorottelevat piikit.
\end{itemize}

\subsection{Renaudin\cite{renaudin2013adaptative}}
\begin{itemize}
  \item Askelluksen analysointi etukäteen ympäristössä, missä GPS-paikannus
    on käytettävissä.
  \item Askeleen pituuden ja tiheyden välille oletetaan malli, jonka parametrit
    estimoidaan GPS-signaalin perusteella.
  \item Kaksi mallia, joista toinen perustuu GPS:n antamaan paikkatietoon ja
    toinen GPS-signaalin doppler-siirtymään.
  \item Doppler-siirtymää on vähemmän herkkä häiriöille.
\end{itemize}

\subsection{Madgwick\cite{madgwick2011estimation}}
\begin{itemize}
  \item Laskennallisesti tehokas menetelmä suunnan laskemiseksi MARG-
    mittausdatasta.
  \item Perinteisillä Kalman-suotimiin perustuvilla menetelmillä on
    useita heikkouksia: ne vaativat korkean näyttöönottotaajuuden, ovat
    monimutkaisia ja vaikeita toteuttaa. Pyörimismekaniikka vaatii
    suuren tilavektorin ja laajennetun Kalman-suotimen ja paljon
    laskentatehoa.
  \item Artikkelissa esitetty menetelmä käyttää kvaternioniesitystä suunnalle.
\end{itemize}
\cite{mautz2012indoor}
\cite{faragher2012opportunistic}

\section{Sanasto}

\glsaddall
\printglossary
\glsaddall
\printglossary[type=\acronymtype,title=Lyhenteet]

\bibliographystyle{plainnat}
\bibliography{kandi}
\end{document}
