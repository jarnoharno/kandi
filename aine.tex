\documentclass[a4paper]{scrartcl}
\usepackage[finnish]{babel}
\usepackage[utf8]{inputenc}
\usepackage{verbatim}
\usepackage[numbers]{natbib}
%\usepackage[xindy,nonumberlist,acronym]{glossaries}
\usepackage{hyperref}

%\newglossaryentry{laskelmasuunnistus}{
  name={laskelmasuunnistus},
  description={dead reckoning}
}
\newglossaryentry{partikkelisuodin}{
  name={partikkelisuodin},
  description={particle filter}
}
\newacronym{rssi}{RSSI}{Received Signal Strength Indication}
\newacronym{pdr}{PDR}{Pedestrian Dead Reckoning}
\newacronym{cekf}{CEKF}{Complementary Extended Kalman Filter}
\newacronym{ert}{ERT}{Embedded Reference Trajectory}
\newacronym{wlan}{WLAN}{Wireless Local Area Network}
\newacronym{mems}{MEMS}{Microelectromechanical System}

%\makeglossaries

\subject{Aine}
\author{Jarno Leppänen}
\title{Sisätilapaikannus matkapuhelimella}
\date{14.3.2014}

\begin{document}

\maketitle

\tableofcontents

\section{Johdanto}

Älypuhelimien ja muiden mobiililaitteiden yleistymisen myötä kiinnostus
luotettavan paikkatiedon hyödyntämiseen on kasvanut\cite{harle2013survey}.
Tämän päivän tyypillinen älypuhelin tarjoaa käyttäjälleen paikalliset uutiset
ja sääennusteen, ohjaa hampurilaistilauksen lähimpään pikaruokalaan, suunnistaa
autoilijan perille ruuhkia välttäen sekä seuraa niin kuntoilijan
kalorikulutusta kuin pyöräilijän ajokilometrejäkin.

Tarkkaan paikannukseen perustuvien sovellusten yleistymisen on mahdollistanut
satelliittipaikannuksen (GNSS) ja erityisesti GPS-järjestelmän käyttöönotto
mobiililaitteissa. Satelliittipaikannus toimii kuitenkin huonosti
sisätiloissa
Sisätiloissa tapahtuvalle paikannukselle on monia sovellusmahdollisuuksia. 


Satelliittipaikannus toimii kuitenkin signaalin
heikkenemisen ja heijastumien takia huonosti sisätiloissa.

Sisätiloissa tapahtuvalle paikannukselle on useita sovelluksia...

Haasteellisuuteen on löydettävissä useita
syitä\cite{mautz2012indoor}:
\begin{itemize}
  \item seinistä ja huonekaluista tapahtuvat heijastukset
  \item suoran näköyhteyden puute lähettimen ja vastaanottimen välillä
  \item signaalin heikkeneminen sisätiloissa olevien esteiden takia
  \item muutokset signaalin kulussa ja voimakkuudessa läsnäolevien ihmisten
    ja ympäristön muutosten takia
  \item sisätiloissa tapahtuvan paikannuksen korkeampi tarkkuusvaatimus
    ulkotilapaikannukseen verrattuna
\end{itemize}

\bibliographystyle{plainnat}
\bibliography{kandi}
\end{document}
