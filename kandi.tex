\documentclass[a4paper]{scrartcl}
\usepackage[finnish]{babel}
\usepackage[utf8]{inputenc}
\usepackage{verbatim}
\usepackage[numbers]{natbib}
\usepackage{hyperref}
\subject{Kandidaatintyö}
\author{Jarno Leppänen}
\title{Laskelmasuunnistus matkapuhelinsensoreilla sisätilapaikannuksessa}
\date{14.2.2014}
\begin{document}
\maketitle

\tableofcontents

\section{Johdanto}

\section{Absoluuttiset paikannustekniikat}
\subsection{GPS}
\cite{groves2013principles}
\subsection{Radiolähteet}
\cite{varshavsky2005gsm}
\subsubsection{Kalibrointi}
\begin{itemize}
\item HORUS
\item EZ
\end{itemize}

\section{Suhteelliset paikannustekniikat}
\subsection{Inertiaalinavigointi}
\begin{itemize}
\item Kumuloituva virhe
\item Virheenkorjaus rajoitusehdoilla
\end{itemize}
\cite{groves2013principles}
\subsection{Jalankulkijan laskelmasuunnistus}
\cite{harle2013survey}
\subsubsection{SHS}
\begin{itemize}
\item Askeleen tunnistus
\item Askeleen pituuden estimointi
\item Suunnan estimointi
\item ZUPT-rajoitusehto
\end{itemize}

\section{Hybriditekniikat}
\cite{woodman2010pedestrian}
\cite{evennou2006advanced}
\cite{leppakoski2013pedestrian}
\subsection{Bayes-suotimet}
\subsubsection{Kalman-suodin}
\subsubsection{Partikkelisuotimet}
\subsection{Karttatieto rajoitusehtona}
\cite{li2012reliable}
\subsection{Radiotunnistetiedon ja laskelmasuunnistuksen yhdistäminen}
\cite{rai2012zee}

\section{Matkapuhelinsovellukset}
\subsection{Absoluuttisen paikannukseen perustuvat navigaattorit}
\subsubsection{Kokeiluasteella olevat järjestelmät}
Bluetooth, RFID, magneettikenttä\ldots
\subsection{Askelmittarit}
Runtastic, jne.
\subsection{Kokeiluasteella oleva hybridijärjestelmät}
Zee?

\section{Yhteenveto}

\bibliographystyle{plainnat}
\bibliography{kandi}
\end{document}
