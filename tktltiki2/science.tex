\section{Johdanto}

TODO

Tässä työssä tarkastellaan älylaitteessa olevien sensoreiden hyödyntämistä
sisätilapaikannuksessa.

Älylaitteella tarkoitetaan tässä kannettavaa verkkoon kytkettyä laitetta, joka
sisältää laskelmasuunnistuksessa tarvittavia sensoreita. Näihin sensoreihin
kuuluu kiihtyvyysanturi, gyroskooppi, magnetometri sekä joskus myös
ilmanpainemittari.

Mautz, haasteet \cite{mautz2012}

\section{Radiosignaaliin perustuva paikannus}

TODO

GPS

Etäisyys ja kulmamittaus

Proximity

\subsection{Radiosormenjälkimenetelmät}

TODO

Radiosormenjälkiin (fingerprinting) perustuvissa menetelmissä kohteena olevasta
sisätilasta kartoitetaan riittävällä tarkkuudella WLAN-tukiasemien voimakuudet
ja ratkaistaan käyttäjän sijainti vertaamalla signaalien voimakkuutta
aiemmin kartoitettuun...

Ongelmana näille menetelmille on kalibrointidatan keräämisen korkea kustannus.

Deterministisessä paikannuksessa sijaintia ei oleteta satunnaismuuttujaksi,
jolloin se voidaan laskea esimerkiksi k:n lähimmän naapurin menetelmällä...

Sijainti voidaan laskea myös olettamalla sijainti satunnaismuuttujaksi
ja laskemalla sen posteriorijakauma sormenjälkimittauksia käyttäen...

Log-etäisyysmalli kuvaa signaalin vaimenemista etäisyyden funktiona...

Tukiasemien lähetysvoimakkuus, signaalin varjostumista kuvaavat parametrit
voidaan ratkaista sormenjälkimittauksista...

\section{Jalankulkijan laskelmasuunnistus}

Jalankulkijan laskelmasuunnistuksella (PDR; engl: Pedestrian Dead Reckoning)
tarkoitetaan käyttäjän sijainnin laskentaa käyttäjän mukana olevan laitteen
sensoreista saatavan tiedon perusteella. Tavallisesti nämä sensorit ovat
kiihtyvyysantureita ja gyroskooppeja, mutta menetelmässä voidaan hyödyntää myös
magnetometreja sekä ilmanpaineantureita.

Laskelmasuunnistuksessa tarkastellaan paikan muutosta edelliseen
mittauspisteeseen verrattuna, joten menetelmä vaatii tiedon käyttäjän
lähtötilasta. Laskelmasuunnistus toimii siis ilman laitteen ulkopuolista
infrastruktuuria, mutta paikannuksen virhe kumuloituu ajan myötä ilman
sijainnin ajoittaista päivitystä absoluuttisen paikannusjärjestelmän
avulla.

\subsection{Inertiaalinavigointi}

Inertiaalinavigoinnissa inertiaalimittausyksikön (IMU; engl: Inertial
Measurement Unit) sijainti lasketaan suhteessa sen alkusijaintiin,
-kulkusuuntaan ja -nopeuteen. IMU sisältää kolme toisiinsa nähden
ortogonaalista kiihtyvyysanturia ja gyroskooppia.
Inertiaalinavigointijärjestelmä (INS; engl: Inertial Navigation System) koostuu
IMU:stä sekä navigointiprosessorista, joka huolehtii sijainnin laskennasta
sensoreista saatavan mittaustiedon perusteella. Useimmiten IMU asennetaan
seurattavaan laitteeseen kiinteästi (strapdown) \cite{harle2013}.

Kiinteässä konfiguraatiossa kulkusuunta lasketaan
integroimalla gyroskoopeista saatava kulmanopeus ajan suhteen ja lisäämällä
tulos edellisen mittauspisteen kulkusuuntaan. Järjestelmän todellinen
kiihtyvyys taas saadaan vähentämällä painovoiman aiheuttama näennäiskiihtyvyys
kiihtyvyysantureiden mittaamasta spesifisestä kiihtyvyydestä ja nopeus
integroimalla todellinen kiihtyvyys ajan suhteen ja lisäämällä tulos edellisen
mittauspisteen nopeuteen. Sijainti saadaan lopuksi integroimalla nopeus ja
lisäämällä edellisen mittauspisteen sijaintiin.

Seurattavan laitteen kiihtyvyyden ja kulmanopeuden on pysyttävä sensorien
mittausalueen rajoissa, mutta muita oletuksia laitteen liikkeen dynamiikasta ei
tehdä, joten paikannuksen virhe kasvaa suhteessa seuranta-ajan kuutioon.
Lentokoneiden, sukellusveneiden ja ohjusten inertinaalinavigointijärjestelmissä
käytettävät erittäin tarkat sensorit mahdollistavat laskelmasuunnisuksen
tuntien aikaväleillä ilman, että paikannuksen virhe kasvaa merkittäväksi.
Älylaitteissa käytettävien mikrosysteemiantureiden (MEMS; engl:
Microelectromechanical Systems) tarkkuus ei tällä hetkellä riitä edellä kuvatun
kaltaisen rajoittamattoman inertiaalinavigaation toteuttamiseen. On lisäksi
huomattava, ettei paraskaan INS voi toimia täysin itsenäisesti ilman
absoluuttisen paikannusjärjestelmän apua painovoiman mikroskooppisesta
vaihtelusta ja laskelmasuunnistuksen kumuloituvasta virheestä johtuen.

Sensoreissa esiintyvien mittausvirheiden vaikutus minimoidaan tavallisesti
hyödyntämällä Kalman-suodinta.
Kalman-suodin on rekursiivinen algoritmi,
jonka avulla voidaan tuottaa optimaalinen estimaatti lineaarisen
fysikaalisen systeemin
tilasta olettaen systeemin tilan prosessi- ja mittausvirheen tilastolliset
ominaisuudet tunnetuiksi. Kalman-suodin voidaan nähdä myös Bayes-suotimen
erikoistapauksena,
jossa tila- ja mittasiirtymäfunktiot ovat lineaarisia ja virheet
normaalijakautuneita. INS-järjestelmän standarditoteutuksessa
käytetään laajenettua epälineaarista Kalman-suodinta, jonka tilamuuttujina
ovat sensoreiden virheet \cite{foxlin2005}.

Kulkusuunnan korjauksessa voidaan hyödyntää magnetometriä, jonka avulla on
mahdollista selvittää laitteen asento maapallon magneettiseen napaan nähden.
Sisätiloissa magnetometrin antamat mittaustulokset vaihtelevat kuitenkin
huomattavasti rakennuksissa olevien metallirakenteiden ja sähkölaitteiden
aiheuttamien häiriöiden vuoksi.

Korkeutta merenpinnasta voidaan arvioida ilmanpainemittarilla. Jalankulkijan
tapauksessa tästä on hyötyä esimerkiksi monikerroksisten rakennusten
sisällä tapahtuvassa paikannuksessa.

\subsubsection{ZUPT}

Inertiaalinavigoinnin kumuloituvaa virhettä voidaan pienentää ottamalla
huomioon järjestelmässä esiintyviä rajoitusehtoja. Jalankulkijan paikannuksessa
sensori voidaan esimerkiksi kiinnittää käyttäjän kantapäähän, jolloin
sensori oletettavasti pysyy paikallaan askelten välissä jalan ollessa maassa.
Tätä rajoitusehtoa kutsutaan nimellä Zero Velocity Update (ZUPT).
ZUPT on yleinen PDR-sovelluksissa ja sitä hyödyntämällä on saavutettu
vain 1--2\% virhe lasketusta matkasta.

\subsection{SHS}

Älylaitteiden MEMS-sensoreilla intertiaalinavigoinnin toteuttaminen ei
nykyisellään ole mahdollista sensorien epätarkkuuden vuoksi.  PDR-sovelluksissa
lupaavimmaksi lähestymistavaksi on osoittaunut askeleen ja kulkusuunnan
tunnistavat menetelmät (SHS; engl: Step-and-Heading System), joissa
jalankulkijan reitti lasketaan kulkusuuntaa ja askelpituutta kuvaavina
askelvektoreina. SHS-järjestelmissä toiminta perustuu kolmeen vaiheeseen:
askelten tunnistamiseen ja askeleen pituuden sekä askeleen kulkusuunnan
arviointiin.

\subsubsection{Askeleen tunnistus}

Askeleen tunnistamisessa hyödynnetään jalankulkijan askelluksen jaksollisuutta.
Useimmiten tunnistus perustuu pystysuunnassa tapahtuvan kiihtyvyysanturidataan,
mutta myös gyroskooppidataa on voidaan hyödyntää. Eräs tapa askeltunnistukseen
on tarkastella sensorin pystysuuntaisessa kiihtyvyydessä esiintyviä huippuja,
jotka aiheutuvat jalan osumisesta maahan joka askeleella \cite{wang2012}.
Toinen tapa on havainnoida pystysuuntaisessa todellisessa kiihtyvyydessä
tapahtuvia nollan ylityksiä, joilla askelluksen jaksot voidaan tunnistaa
\cite{fadjukoff2013}. Askeleet voidaan havaita myös pystysuuntaisen
kiihtyvyysdatan liukuvan varianssin huippukohtina \cite{steinhoff2010}.
Kiihtyvyysdatan liukuvaa autokorrelaatiosarjaa on mahdollista hyödyntää
askelluksen jaksollisuuden havainnointiin. Esimerkiksi Rai et al.
\cite{rai2012} hyödyntävät tätä menetelmää Zee-järjestelmässään.

Käyttäjän mukana kulkeva älylaite, joka ei ole kiinteässä asennossa
tai paikassa suhteessa käyttäjään, asettaa lisähaasteita sensoritiedon
hyödyntämiselle. Esimerkiksi älylaitteen tuottama kiihtyvysanturidata
on hyvin erilaista riippuen siitä, onko laite käyttäjän kädessä vai
taskussa. Tässä työssä tutkituissa tapauksissa laite oli aina kiinteässä
paikassa käyttäjän yllä, yleensä taskussa.

\subsubsection{Askelpituus}

Yksinkertaisin tapa askelpituuden arviointiin on olettaa pituus tai
sen muutos vakioksi. Esimerkiksi Woodman \cite{woodman2008} oletti
askelpituuden muutoksen askelten välillä normaalijakautuneeksi siten,
että muutoksen suuruus on nolla ja keskihajonta vakio. Askelpituus
ei kuitenkaan tyypillisesti pysy vakiona, vaan riippuu muun muassa
askeltiheydestä. Li \cite{li2012} sovitti käyttäjäkohtaisen lineaarisen
mallin askeltiheyden ja -pituuden välille. Renaudin
\cite{renaudin2013} oletti askelpituuden, käyttäjän oman pituuden sekä
askeltiheyden välille mallin, jonka parametrit estimoitiin etukäteen
tarkan satelliittipaikannuksen ja askelmittarin avulla.

Inertiaalinavigaatiota voi käyttää SHS-järjestelmän osana sensorin radan
laskentaan askeleen aikana. Tällaista SHS-INS-menetelmää käytetään,
jos inertiaalinavigaatioon halutaan helposti sisällyttää korkeamman
tason rajoitusehtoja, kuten karttatietoa \cite{woodman2008}. Menetelmä
toimii parhaiten, jos sensori on asennettu käyttäjän jalkaan ja
ZUPT-rajoitusehtoa on mahdollista käyttää.

\subsubsection{Kulkusuunta}

Kulkusuunnan muutoksen voi ratkaista integroimalla askeleen aikana
gyroskoopilta kerättyä kulmanopeusdataa samalla tavalla kuin pelkässä
inertiaalinavigoinnissa. Tämä toimii hyvin sovelluksissa, joissa
laskelmasuunnistukseen käytetty sensori on kiinteässä asennossa suhteessa
käyttäjään. Älylaitteen asento suhteessa käyttäjän kulkusuuntaan voi
kuitenkin vaihdella: laite voi olla taskussa vaikkapa ylösalaisin.

Vapaasti sijoitettavan sensorin tapauksessa kulkusuuntaa voidaan arvioida
kiihtyvyysanturin tuottaman aikasarjan pääkomponenttianalyysin (PCA) avulla.
Jalankulkijan kävellessä liikkeen tuottaman kiihtyvyysdatan suurin
pääkomponentti vastaa kulkusuunnan suuntaista suoraa \cite{steinhoff2010}.
Kulkusuunnan suuntaisen kiihtyvyyden fourier-muunnoksessa havaittavan
perustaajuuden ensimmäinen harmoninen moninkerta on lisäksi voimakkain
taajuus riippumatta sensorin sijainnista suhteessa käyttäjään
\cite{rai2012}.
Pääkomponentti ei
kuitenkaan kerro, kumpaan suuntaan liike suuntautuu, joten suunnan
selvittämiseksi on käytettävä lisäheuristiikkaa. Steinhoff
\cite{steinhoff2010} kokeili useita menetelmiä suunnan estimoimiseksi
ja saavutti taskussa olevalla älylaitteella suuntaestimaatille parhaimmillaan
5.7 \% mediaanivirheen. Menetelmässä kiihtyvyysdata projisoitiin lattian
suuntaiseen tasoon ja käsiteltiin alipäästösuotimella ennen
pääkomponenttianalyysia. Tutkimuksessa liikkeen suunta ratkaistiin
tarkastelemalla sensorin pyörimissuuntaa ennen jalan osumista maahan.

Sensorin absoluuttinen suunta saadaan laskettua magnetometristä, jos paikallisen
magneettikentän suunta tunnetaan. Sisätiloissa magneettikentän suunta
kuitenkin vaihtelee huomattavasti. Rai \cite{rai2012} käytti kulkusuunnan
arvioimiseen magnetometriä, kiinteää etukäteisarviota sisätilan
magneettikentän suunnasta sekä kulun aikana vakiona pysyvää arviota
sensorin asennosta suhteessa käyttäjän kulkusuuntaan. Sensorin asento
laskettiin kiihtyvyysanturidatan perusteella. Yhdistettynä sensorin
magentometrin antamaan tietoon voitiin ilman ylimääräistä
heuristiikkaa selvittää, kumpaan suuntaan käyttäjän
liike suuntautuu.

\subsection{Karttatiedon käyttäminen rajoitusehtona}

Inertiaalinavigaation ja SHS-järjestelmän merkittävänä haittapuolena on
paikannuksen virheen kumuloituminen. Pitkän aikavälin laskelmausuunnistuksen
virhettä voidaan pienentää ottamalla ympäristössä olevia rajoituksia
huomioon. Esimerkiksi autonavigaatiojärjestelmissä auton sijainti pakotetaan
tielle, vaikka GPS-paikantimen sijaintiestimaatti olisikin tien sivussa.
Menetelmää kutsutaan nimellä karttasovitus (Map Matching) ja siinä
tieverkosto esitetään yleensä graafina. Sisätiloissa jalankulkijan reittiä
rajoittavia seiniä ja muita esteitä voidaan niin ikään hyödyntää paikannuksessa
estämällä paikannusalgoritmin tuottamat mahdottomat reitit.

Toisin kuin graafeina kuvatuissa autokartoissa, sisätilakartoissa esteet
esitetään yleensä janoina, joiden läpi paikannusalgoritmin tuottama käyttäjän
reitti ei voi kulkea. Sisätilakartat eivät toistaiseksi ole yleistyneet
eikä vakiintunutta formaattia ole olemassa.

\subsubsection{Partikkelisuotimet}

Suosituin menetelmä karttatietojen hyödyntämiseen sisätilapaikannuksessa
SHS-menetelmän yhteydessä
on käyttää partikkelisuotimia (PF; engl: Particle Filter).
Partikkelisuodin on Bayes-suotimen numeerinen approksimaatio, jossa
tilan todennäköisyysjakauma esitetään partikkelijoukkona
\begin{align}
\mathcal{S}_t &= \{\langle\mathbf{x}_t^i,w_t^i\rangle \ i = 1,\ldots,N\},
\end{align}
missä $\mathbf{x}_t^i$ on partikkelin $i$ tila ja $w_t^i$ sen painokerroin
\cite{woodman2010}.
Partikkelin painokerroin edustaa todennäköisyyttä, jolla tila vastaa
systeemin todellista tilaa.

Partikkelisuotimen tilan päivitykseen on kehitetty useita algoritmeja,
joista suosituin on tärkeysotantaan perustuva SIR
(Sampling-Importance-Resampling). SIR jakautuu kolmeen vaiheeseen.
Ensin partikkelien tila $\mathbf{x}_{t-1}^i$ uudelleennäytteistetään
partikkelijoukon $\mathcal{S}_{t-1}$
määräämästä priorijakaumasta. Tämän jälkeen partikkelien tilat
$\mathbf{x}_t^i$ päivitetään tilamallin jakauman
$p(\mathbf{x}_t | \mathbf{x}_{t-1}^i)$ perusteella. Lopuksi 
partikkelien painot $w_t^i$ päivitetään mittausmallin jakauman
$p(\mathbf{z}_t | \mathbf{x}_t^i)$ perusteella, missä
$\mathbf{x}_t$ on mittaus ajanhetkellä $t$.

Uudelleennäytteistyksen tarkoituksena on poistaa partikkelipilvestä
partikkelit, joiden paino on pieni ja toisaalta jakaa suuripainoiset
partikkelit joukoksi pienempipainoisia partikkeleita.
Uudelleennäytteistysalgoritmi pyrkii valitsemaan uudet partikkelit siten,
että partikkelipilvi kuvaa mahdollisimman hyvin tilajakaumaa.

Uudelleennäytteistykseen on kehitetty useita algoritmeja.
Woodman \cite{woodman2008} käytti uudelleennäytteistykseen
Kullback-Leibler-divergenssiin perustuvaa menetelmää. Joissain sovelluksissa
uudelleennäytteistystä ei suoriteta joka askeleella, vaan ainoastaan jos
esimerkiksi efektiivinen näytteitten määrä (ESS; engl: effective sample size)
$\hat{N}_{\text{eff}} = 1 / \sum_{i=1}^N (w_k^i)^2$ alittaa tietyn
raja-arvon \cite{fadjukoff2013}.

SHS-järjestelmissä systeemin tilamuuttujina käytetään tyypillisesti
kolmikkoa $\mathbf{x}_t = (x_t, y_t, \theta_t)$, missä $(x_t, y_t)$ kuvaa
käyttäjän sijaintia koordinaatistossa ajanhetkellä $t$ ja $\theta_t$
käyttäjän kulkusuuntaa. 
Woodmanin \cite{woodman2008} järjestelmässä tilan päivitys tapahtui
seuraavan mallin mukaisesti:
\begin{align}
  x_{t+1}       &= x_t + (l_t + n_t^l) \cos(\delta\theta_t + n_t^\theta) \\
  y_{t+1}       &= y_t + (l_t + n_t^l) \sin(\delta\theta_t + n_t^\theta) \\
  \theta_{t+1}  &= \theta_t + \delta\theta_t + n_t^\theta,
\end{align}
missä kaksikko $(l_t, \delta\theta_t)$ on askelluksen aikana toimivan
INS-järjestelmän tuottama askeleen pituus ja kulkusuunnan muutos vastaavasti
ajanhetkellä $t$. Kaksikko $(n_t^l, n_t^\theta)$ taas on näiden suureiden
virhejakaumista generoitu satunnaislukupari.

Karttatieto huomioidaan partikkelisuotimessa painotusvaiheessa. Yksinkertaisin
ratkaisu on asettaa partikkelin painokerroin nollaksi, jos partikkelin
kulkema jana leikkaa kartalla olevan seinän kanssa. Karttatiedoissa
mahdollisesti esiintyvät virheet voidaan huomioida asettamalla
uudeksi painoksi jokin edellisen painon murto-osa, jos reitti leikkaa
esteen läpi \cite{fadjukoff2013}.

Partikkelisuotimet mahdollistavat mutkikkaampienkin heuristiikkojen
käytön joustavasti. Rain \cite{rai2012} Zee-järjestelmä esimerkiksi
tallensi partikkelijoukon kulkeman reitin ja eliminoi tulevaisuudessa
mahdottomaksi osoittautuneita reittejä backpropagation-algoritmilla, jolloin
koko käyttäjän kulkema reitti pystyttiin ratkaisemaan.

Partikkelisuotimet vaativat tyypillisesti runsaasti tallennuskapasiteettia
ja laskentatehoa. Vaadittava partikkelimäärä tietyllä ajanhetkellä 
on verrannollinen
tilan todennäköisyysjakaumaan: mitä epävarmempi käyttäjän tila on, sitä
useampi partikkeli tarvitaan edustamaan tilan todennäköisyysjakaumaa.
Tilavaativuus kasvaa lisäksi eksponentiaalisesti
Tilamuuttujavektorin dimensionaalisuuden kasvaessa.

Alkutilassa partikkeleita tarvitaan eniten, jos käyttäjän sijaintia ei
voida määrittää etukäteen. Alustukseen voi käyttää esimerkiksi
GPS- tai WiFi-paikannusta \cite{woodman2009}. Woodman \cite{woodman2010}
tunnisti kaksi tyypillistä
vaihetta partikkelisuotimen laskennassa. Paikannusvaiheessa epävarmuus
on suurta ja partikkelimäärä korkea. Seurantavaiheessa taas, jossa käyttäjän
on jo pystytty paikantamaan, tilatodennäköisyysjakauma on keskittynyt, eikä
partikkeleita tarvita niin montaa. Woodmanin tutkimuksessa 8725
$\text{m}^2$ kokoisessa kolmikerroksisessa rakennuksessa täydellisen epävarman
priorijakauman kuvaamiseen tarvittiin 4 miljoonaa partikkelia, kun taas
seurantavaiheessa vain 500 partikkelia riitti kuvaamaan
tilatodennäköisyysjakaumaa. Laskennan vaativuus asettaa merkittäviä
haasteita, jos partikkelisuotimia halutaan hyödyntää älylaitteiden
paikannuksessa.

\section{Hybridijärjestelmät}

Radiosignaalipaikannuksessa käytettävien sormenjälkimenetelmien suurimpana
haittapuolena on kalibrointidatan keräämiseen liittyvät ongelmat.
Radiosormenjälkiaineistoa tarvitaan paljon riittävän tarkkuuden saavuttamiseksi.
Lisäksi aineston täytyy pysyä ajan tasalla
ympäristössä olevien kiinteiden esteiden siirtyessä ja signaalilähteiden
muuttuessa.

PDR-menetelmät voivat helpottaa radiokartoitusta automatisoimalla osan
kartoituksen vaatimasta paikannuksesta. Woodman ja Harle \cite{woodman2008}
hyödynsivät
kenkään kiinnitettyä sensoria radiokartioituksen tekemiseen. Järjestelmä
käytti laskelmasuunnistusta paikannukseen kartoittajan kävellessä
ympäri rakennusta ja otti samalla radiosormenjälkinäytteitä. 
Muodostettua radiokarttaa hyödynnettiin partikkelisuotimessa
eliminoimalla tilat, joissa partikkeli ei voinut radiosormenjäljen
perusteella olla.

WiFi-paikannus ja PDR toimivat optimaalisesti erilaisissa olosuhteissa
ja usein täydentävät toisiaan.
Leppäkoski et al. \cite{leppakoski2013} laativat
järjestelmän, jossa
sormenjälkiin perustuvaa paikannusta hyödynnettiin yhdessä 
vyöhön kiinnitettyyn sensoriin perustuvan PDR-menetelmän
kanssa laajennetulla Kalman-suotimella. Tutkimuksessa todettiin,
että WiFi-paikannus tukee PDR-paikannusta sisätilan alueilla, joissa
karttarajoitteita ei ole paljon. Toisaalta sormenjäljillä voidaan
eliminoida rakennuksen eri osissa olevia partikkeleita.

Hybridilähestymistapaa voi hyödyntää myös käyttämällä PDR-menetelmän avulla
kerättyä radiokarttaa suoraan WiFi-paikannukseen. Lupaavassa
tutkimuksessaan Rai et al. \cite{rai2012} joukkoistivat radiokartoituksen
käyttäjille, joiden älylaitteet tukivat PDR-menetelmää. Järjestelmä
kykeni lisäksi päivittämään radiokarttaa jatkuvasti hödyntämään PDR-menetelmää
valikoivasti riippuen käyttäjän älylaitteen ominaisuuksista. Tutkimuksessa
kuvataan 15 tunnin koe yhdellä käyttäjällä, jossa muutaman tunnin aikana
automaattisesti kerättyä radiokarttaa hyödynnetään onnistuneesti
WiFi-paikannukseen. 

Faragher et. al \cite{faragher2012} kuvasivat järjestelmän, joka ei
tukeutunut lainkaan keskitettyyn radio- tai sisätilakarttaan. Järjestelmä
käytti kiihtyvyysanturi- ja magnetometridataa SHS-laskelmasuunnistukseen ja
kartoitti samaan aikaan radiosormenjälkiä. Sormenjälkien samankaltaisuuteen
perustuen järjestelmä havaitsi syklejä käyttäjän kulkemissa reiteissä
ja kykeni näin muodostamaan sisätilassa havaituista alueista
verkoston, jota voitiin käyttää navigointiin. Applen ostama WiFiSLAM
on samankaltaiseen järjestelmään perustuva kaupallinen yritys.

\section{Yhteenveto}

Älylaitteissa yleistyneet MEMS-sensorit eivät ole riittävän tarkkoja
inertiaalinavigaation toteuttamiseen. Tarkkoja tuloksia
pelkällä inertiaalinavigaatiojäjrjestelmällä
on saatu aikaan
ainoastaan kenkään kiinnitetyn IMU:n tapauksessa.

Vapaasti käyttäjän yllä olevilla laitteilla
karttatietoa hyödyntäviin partikkelisuotimiin yhdistetyt SHS-järjestelmät ovat
osoittautuneet lupaaviksi ratkaisuiksi.
Nämä järjestelmät pystyvät parhaimmillaan seuraamaan jatkuvasti liikkeessä
olevaa käyttäjää virheen kumuloitumatta, jos käyttäjä kulkee alueilla, joissa
karttatiedon rajoitteita voi hyödyntää. Hetket, jolloin käyttäjä on paikallaan
ovat osoittauneet ongelmallisiksi virheellisten askellusten myötä syntyvän
paikannusvirheen takia.

Sisätiloissa oleileva henkilö tyypillisesti pysyy osan ajasta paikallaan
ja vaihtaa välillä paikkaa kävellen. Näissä tiloissa kunnollisesti toimiva
SHS-paikannus edellyttää käyttäjän modaliteetin
tunnistuksen yhdistämistä paikannusjärjestelmään.
Järjestelmän on kyettävä tunnistamaan, onko käyttäjä kävelemässä vai paikallaan.
Omat haasteensa tälle asettaa älylaitteen vapaa sijoittelu: käyttäjä voi
pitää älylaitetta taskussaan, korvallaan, edessään tai vaikkapa pöydällä
laitteeseen välillä tarttuen. Hemminki et al. \cite{hemminki2013} tunnistavat
luokittelumenetelmiä hyödyntäen ajoneuvotyypin, jossa käyttäjä matkustaa
tietyllä ajanhetkellä. Tutkimuksessa älylaitteen kiihtyvyysanturidatasta
tunnistetaan piirteitä, joiden perusteella luokittelu tapahtuu. Suomalaisessa
kaupallisessa Moves-älypuhelinsovelluksessa modaliteetin tunnistus yhdistetään
paikannukseen, automaattiseen paikan tunnistukseen (engl: Place Detection)
sekä askelmittariin. Menetelmä hyödyntää epätarkkaan välittömään tunnistamiseen
älylaitteen paikallisia resursseja ja analysoi sensoridatan tarkemmin
palvelimella korjaten aiemmassa analyysissa tapahtuneita virheitä.

PDR-järjestelmän ja radiosignaaliin perustuvan paikannuksen yhdistämiseen on
kehitetty monia menetelmiä. Lupaavimmat järjestelmät ovat yhdistäneet
Wifi-radiokartoituksen ja WiFi-paikannuksen PDR-menetelmiin.
Nämä menetelmät tukevat toisiaan: PDR-menetelmissä tapahtuvaa virheen
kumuloitumista voidaan kompensoida WiFi-paikannuksella ja toisaalta
PDR-menetelmää hyödyntävällä joukkoistamisella voidaan helpottaa
aikaavievää radiokartoitusta. PDR-menetelmillä on myös mahdollista paikata
radiokartoituksessa olevia aukkoja, joissa sormenjälkiä on vähän tai 
tukiasemien kuuluvuus on rajoitettu.

Näiden innovaatioiden hyödyntäminen ei rajoitu pelkästään WiFi-verkkoihin,
vaikka suurin osa tähän astisesta tutkimuksesta on keskittynyt
nimenomaan WiFi-paikannuksen ja PDR-menetelmien yhdistämiseen. Esimerkiksi
lukuetäisyydeltään rajoitettujen RFID-tunnisteiden ja -lukijoiden avulla
voidaan pienin kustannuksin kattaa suuri osa sisätilasta. PDR-menetelmällä
pystyttäisiin hyvin kattamaan katvealueet, joissa RFID-tunnisteisiin
perustuva paikannus ei toimi. Toinen lähitulevaisuudessa yleistyvä lupaava
teknologia on matalaenergiset Bluetooth-laitteet (BLE; Bluetooth Low Energy).
BLE-majakat ovat erittäin edullisia, toimivat yhdellä latauksella jopa
vuosia ja mahdollistavat suurten aluiden kattamisen pienellä investoinnilla.
Useimmat uudet älylaitteet tukevat BLE-standardia, joten teknologian
hyödyntäminen olisi mahdollista suuressa mittakaavassa jo nyt.

Vaikuttaa todennäköiseltä, että tulevaisuudessa tullaan näkemään
järjestelmiä, jotka hyödyntävät erilaisia radioverkkoja
tilannekohtaisesti absoluuttiseen paikannukseen. Kehittyvät PDR- ja
hybridimenetelmät yhdessä uusien verkkojen kanssa voivat osoittautua
ratkaisuiksi tarkkaan sisätilapaikannukseen.
