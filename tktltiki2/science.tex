\section{Johdanto}

Tässä työssä tarkastellaan älylaitteessa olevien sensoreiden hyödyntämistä
sisätilapaikannuksessa. 

Siä

Älylaitteella tarkoitetaan tässä kannettavaa verkkoon kytkettyä laitetta, joka
sisältää laskelmasuunnistuksessa tarvittavia sensoreita. Näihin sensoreihin
kuuluu kiihtyvyysanturi, gyroskooppi, magnetometri sekä joskus myös
ilmanpainemittari.

Mautz, haasteet \cite{mautz2012indoor}


\section{Radiosignaaliin perustuva paikannus}

GPS

Etäisyys ja kulmamittaus

Proximity

Radiosormenjälkiin (fingerprinting) perustuvissa menetelmissä kohteena olevasta
sisätilasta kartoitetaan riittävällä tarkkuudella WLAN-tukiasemien voimakuudet
ja ratkaistaan käyttäjän sijainti vertaamalla signaalien voimakkuutta
aiemmin kartoitettuun...

Ongelmana näille menetelmille on kalibrointidatan keräämisen korkea kustannus.

Deterministisessä paikannuksessa sijaintia ei oleteta satunnaismuuttujaksi,
jolloin se voidaan laskea esimerkiksi k:n lähimmän naapurin menetelmällä...

Sijainti voidaan laskea myös olettamalla sijainti satunnaismuuttujaksi
ja laskemalla sen posteriorijakauma sormenjälkimittauksia käyttäen...

Log-etäisyysmalli kuvaa signaalin vaimenemista etäisyyden funktiona...

Tukiasemien lähetysvoimakkuus, signaalin varjostumista kuvaavat parametrit
voidaan ratkaista sormenjälkimittauksista...

\section{Jalankulkijan laskelmasuunnistus}

Jalankulkijan laskelmasuunnistuksella (PDR; engl: Pedestrian Dead Reckoning)
tarkoitetaan käyttäjän sijainnin laskentaa käyttäjän mukana olevan laitteen
sensoreista saatavan tiedon perusteella. Tavallisesti nämä sensorit ovat
kiihtyvyysantureita ja gyroskooppeja, mutta menetelmässä voidaan hyödyntää myös
magnetometreja sekä ilmanpaineantureita.

Laskelmasuunnistuksessa tarkastellaan paikan muutosta edelliseen
mittauspisteeseen verrattuna, joten menetelmä vaatii tiedon käyttäjän
lähtötilasta. Laskelmasuunnistus toimii siis ilman laitteen ulkopuolista
infrastruktuuria, mutta paikannuksen virhe kumuloituu ajan myötä ilman
sijainnin ajoittaista päivitystä absoluuttisen paikannusjärjestelmän
avulla.

\subsection{Inertiaalinavigointi}

Inertiaalinavigoinnissa inertiaalimittausyksikön (IMU; engl: Inertial
Measurement Unit) sijainti lasketaan suhteessa sen alkusijaintiin,
-kulkusuuntaan ja -nopeuteen. IMU sisältää kolme toisiinsa nähden
ortogonaalista kiihtyvyysanturia ja gyroskooppia.
Inertiaalinavigointijärjestelmä (INS; engl: Inertial Navigation System) koostuu
IMU:stä sekä navigointiprosessorista, joka huolehtii sijainnin laskennasta
sensoreista saatavan mittaustiedon perusteella. Kulkusuunta lasketaan
integroimalla gyroskoopeista saatava kulmanopeus ajan suhteen ja lisäämällä
tulos edellisen mittauspisteen kulkusuuntaan. Järjestelmän todellinen
kiihtyvyys taas saadaan vähentämällä painovoiman aiheuttama näennäiskiihtyvyys
kiihtyvyysantureiden mittaamasta spesifisestä kiihtyvyydestä ja nopeus
integroimalla todellinen kiihtyvyys ajan suhteen ja lisäämällä tulos edellisen
mittauspisteen nopeuteen. Sijainti saadaan lopuksi intgeroimalla nopeus ja
lisäämällä edellisen mittauspisteen sijaintiin.

Seurattavan laitteen kiihtyvyyden ja kulmanopeuden on pysyttävä sensorien
mittausalueen rajoissa, mutta muita oletuksia laitteen liikkeen dynamiikasta ei
tehdä, joten paikannuksen virhe kasvaa suhteessa seuranta-ajan kuutioon.
Edellä kuvatulla tavalla toteutettuja inertiaalinavigointijärjestelmiä on
käytössä lentokoneissa, ohjuksissa ja sukellusveneissä, mutta älylaitteissa
käytettävien mikrosysteemiantureiden (MEMS; engl: Microelectromechanical
Systems) tarkkuus ei tällä hetkellä riitä rajoittamattoman
inertiaalinavigaation toteuttamiseen. On lisäksi huomattava, ettei
paraskaan INS voi toimia täysin itsenäisesti ilman absoluuttisen
paikannusjärjestelmän apua painovoiman mikroskooppisesta vaihtelusta ja
kumuloituvasta virheestä johtuen.

Sensoreissa esiintyvien mittausvirheiden vaikutus minimoidaan tavallisesti
hyödyntämällä Kalman-suodinta. Kalman-suodin on rekursiivinen algoritmi,
jonka avulla voidaan tuottaa optimaalinen estimaatti fysikaalisen systeemin
tilasta olettaen systeemin tilan prosessi- ja mittausvirheen tilastolliset
ominaisuudet tunnetuiksi.

Kulkusuunnan korjauksessa voi hyödyntää magnetometriä, jonka avulla voidaan
selvittää laitteen asento maapallon magneettiseen napaan nähden. Sisätiloissa
magnetometrin antamat mittaustulokset vaihtelevat kuitenkin huomattavasti
rakennuksissa olevien metallirakenteiden ja sähkölaitteiden aiheuttamien
häiriöiden vuoksi.

Korkeutta merenpinnasta voidaan arvioida ilmanpainemittarilla. Jalankulkijan
tapauksessa tästä on hyötyä esimerkiksi monikerroksisten rakennusten
sisällä tapahtuvassa paikannuksessa.




Inertiaalinavigoinnin tarkkuutta 




Kalman-suodi

ZUPT

\subsection{SHS}

Askeleen tunnistus

Steinhoffin järjestelmässä \cite{steinhoff2010dead} taskussa olevan
kiihtyyvyysanturin painovoiman suuntaisesta varianssista tunnistettiin
askeleet

Renaudin olettaa käyttäjän askelpituuden ja askeltiheyden välille mallin,
jonka parametrit estimoidaan tilanteessa, missä GPS-paikannus toimii.

\subsection{Kulkusuunta}

Käyttäjän mukana oleva älylaite voi olla missä asennossa tahansa suhteessa
käyttäjän kulkusuuntaan. Laite voi lisäksi kulkea mukana esimerkiksi
taskukssa, laukussa tai kädessä.

Kunze \cuite{kunze2009way} pääkomponenttianalyysi

Steinhoff laski kulkusuunnan taskussa reittä vasten olevan sensorin
kulmamuutoksista ja PCA...

Madgwick \cite{madgwick2011estimation} kehitti menetelmän kulkusuunnan
laskemiseksi MARG-mittausdatasta.
Menetelmässä suunta ratkaistaan minimoimalla kvaternioniesitys...

Zee-järjestelmässä \cite{rai2012zee} käyttäjän suunta tunnistettiin 3-akselisen
kiihtyvyysanturin tuottaman aikasarjan pääkomponenttianalyysistä sekä
autokorrelaatiosta. Liikkeen suunnassa perustaajuus on voimakkaimmillaan
ja liikettä vastaan kohtisuorassa suunnassa ensimmäinen yläsäveltajuus
on voimakkain...

\subsection{Karttatiedon käyttäminen rajoitusehtona}

Sisätilassa olevia esteitä voidaan hyödyntää laskelmasuunnistuksessa
estämään mahdottomat trajektorit.

Partikkelisuotimessa käyttäjän tila
esitetään partikkelien muodostamana todennäköisyyspilvenä...

Suodin on iteratiivinen ja sisältää seuraavat vaiheet: päivitys, korjaus,
uudelleennäyteistys

Woodman ja Harle käyttävät menetelmässään \cite{woodman2009rf} priorijakaumassa
4 miljoonaa partikkelia...

\section{Hybridijärjestelmät}

Suurin este sormenjälkimenetelmien hyödyntämisessä on radiokartoituksen
suorittaminen ja ylläpito. Laskelmasuunnistusta voidaan käyttää tässä apuna...

Woodman ja Harle hyödynsivät kenkään kiinnitettyä sensoria radiokartoituksen
tekemiseen...

Rain zee kerää paikannusjärjestelmän käyttäjiltä samanaikaisesti
radiosormenjälkidataa...

Faragher \cite{faragher2012opportunistic} kerää käyttäjiltä radiosormenjälkien
ja inertiaalidatan lisäksi GPS-dataa

Leppäkoski \cite{leppakoski2013pedestrian} käytti laajennettua Kalman-suodinta
ja vyöhön kiinnitettyä sensoria laskelmasuunnistuksen ja
wlan-paikannuksen yhdistämiseen...

\section{Yhteenveto}

Älylaitteissa yleistyneet MEMS-sensorit eivät ole riittävän tarkkoja
inertiaalinavigaation toteuttamiseen. Hyviä tuloksia on saatu aikaan
ainoastaan kenkään kiinnitetyllä IMU:lla, jolloin sensorin asentoa ja liikettä
voidaan rajoittaa (ZUPT). Vapaasti käyttäjän yllä olevilla laitteilla
SHS-menetelmät ovat osoittautuneet lupaaviksi.

Sisätilojen karttatietoa partikkelisuotimella hyödyntävät järjestelmät
kykenevät parhaimmillaan alle metrin paikannustarkkuuteen...

Haasteena on älylaitteen sijainti käyttäjän yllä ja käyttäjän erilaiset
toimet paikannuksen aikana. Järjestelmät toimivat hyvin, jos käyttäjä
on jatkuvassa liikkeessä...

Suomalainen kaupallisessa käytössä oleva
Moves-älypuhelinsovellus tunnistaa käyttäjän modaliteetin, paikan ja
sisältää tarkan askelmittarin. Esimerkiksi askelluksen tunnistus suoritetaan
aluksi epätarkasti puhelimella, mutta myöhemmin palvelimella analysoidaan
sensoridataa ja tarkennetaan estimaattia kuljettujen askelten määrästä.

Hemminki et. al hyödyntää kiihtyvyysanturidataa käyttäjän modaliteetin
tunnistamiseen ja tunnistaa, jos käyttäjä istuu autossa tai kävelee tjsp.

Radiosignaaliin perustuvien absoluuttisten paikannusjärjestelmä
ja laskelmasuunnistus voidaan yhdistää monella tavalla. Lupaavimmiksi
ovat osoittautuneet järjestelmät, joissa sormenjälkiaineistoa täydennetään
käyttäjien laskelmasuunnistusta hyödyntäviltä älylaitteilta ja 
toisaalta absoluuttiselta paikannusjärjestelmältä hödynnetään
opportunistisesti laskelmasuunnistuksen yhteydessä. Laskelmasuunnistus ja
WiFi-paikannus täydentävät näin toisiaan: 

Radiosormenjälkiä voidaan hyödyntää WiFi-järjestelmien lisäksi myös
GSM- ja Bluetooth-verkoissa sekä RFID-tunnisteissa. Älylaitteissa yleistyvä
matalaenerginen
Bluetooth-standardi ja tätä teknologiaa hyödyntävät edulliset bluetooth-majakat
saattavat lähitulevaisuudessa olla merkittävä ... WifiSLAM

