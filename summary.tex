\documentclass[a4paper]{scrartcl}
\usepackage[finnish]{babel}
\usepackage[utf8]{inputenc}
\usepackage{verbatim}
\usepackage[numbers]{natbib}
\usepackage{hyperref}
\subject{Artikkelireferaatti}
\author{Jarno Leppänen}
\title{Matkapuhelinverkkoon perustuva paikannus
paikkatietoa käyttävien sovellusten tarpeisiin}
\date{31.1.2014}
\begin{document}
\maketitle

\section{Johdanto}

Varshavsky et al.\cite{Varshavsky06aregsm} esittävät
matkapuhelinverkkoihin perustuvan paikannuksen olevan
varteenotettava ratkaisu paikkatietoa hyödyntävien
sovellusten tarpeisiin. Artikkelin mukaan sovellukset
hyödyntävät paikkatietoa tyypillisesti kolmella eri tavalla:
sisätiloissa tapahtuvaan paikannukseen, ulkotiloissa
tapahtuvaan paikannukseen sekä merkityksellisten paikkojen
tunnistamiseen. Nämä tarpeet ovat tyypillisesti erilaisia:
sisätiloissa korostuu paikannuksen tarkkuus ja ulkotiloissa
paikannuksen kattavuus.

\section{Taustaa}

Yleisin paikannustekniikka on satelliittipaikannukseen
perustuva GPS, joka on erinomainen avoimissa ulkotiloissa,
mutta toimii huonommin sisä- ja kaupunkitiloissa.

WiFi-paikannus on WiFi-verkkojen yleisyyden vuoksi
houkutteleva vaihtoehto paikannukseen.  Artikkelissa
esitetään useita hankkeita, joissa WiFi-paikannuksen
tarkkuudesta on saatu lupaavia tuloksia.  Suuren
virrankulutuksen vuoksi WiFi on kuitenkin sitä tukevissa
laitteissa usein kytketty pois päältä, mikä heikentää
mahdollisuuksia WiFin hyödyntämiseen jatkuvassa
paikannuksessa.

Artikkelin mukaan matkapuhelintekniikalla on useita hyviä
puolia paikannusta hyödyn\-tävien sovellusten näkökulmasta.
Matkapuhelimet ovat aina mukana, ne ovat jatkuvasti kiinni
verkossa ja niissä käytetään vakiintunutta tekniikkaa.
Matkapuhelinverkot toimivat tarkkaan säädellyillä
taajuusalueilla, joten ne kärsivät WiFi-verkkoa vähemmän
erilaisista häiriölähteistä.  Matkapuhelinverkoilla on
lisäksi WiFi-verkkoja laajempi kattavuus ja verkkoympäristö
muuttuu verkkojen rakentamiseen vaadittavien suurten
investointikustannusten vuoksi hitaasti.

\section{GSM}

GSM on laajalle levinnyt matkapuhelinverkkostandardi:
GSM-verkkoja on artikkelin kirjoittamisen aikaan ollut 210
maassa 676 operaattorin ylläpitämänä.

GSM-verkko jakautuu maantieteellisesti
matkapuhelintukiasemien virittämiin soluihin. Kullakin
solulla on tietty kontrollikanava, jolla solut lähettävät
vakioteholla signaalia muun muassa naapurisolujen
asetuksista. Eri solujen lähettämien signaalien avulla
matkapuhelinverkon käyttäjä voi vertailla solujen
kuuluvuutta ja vaihtaa tarvittaessa voimakkaamman signaalin
tarjoavaan soluun.

Artikkelissa esiteltiin kahdelle matkapuhelinverkossa
toimivalle laitteelle kehitetty ohjelmisto, jonka avulla
laitteiden soluilta keräämään signaalidataan päästään
käsiksi. Signaalidata sisältää tiedon solun tunnisteesta ja
solun lähettämän signaalin voimakkuudesta.

\section{Paikannus}

Artikkelissa käytettiin \texttt{fingerprinting}- ja
\texttt{centroid}-menetelmää paikkatiedon laskemiseen GSM-
ja WiFi-verkoissa. Molemmat menetelmät vaativat
opetusvaiheen, jossa tarkkaan paikkatietoon yhdistetyn
signaalidatan avulla laaditaan malli, jolla paikkatieto
voidaan laskea vastaanotetun uuden signaalidatan
perusteella. Opetusvaiheessa vaadittava tarkka paikkatieto
saadaan GPS-signaalista.

\texttt{Centroid}-menetelmässä kerätyn tiedon perusteella
lasketaan solujen tukiasemien sijainnit, jolloin uuden
signaalitiedon perusteella voidaan estimoida
matkapuhelinverkon käyttäjän sijainti. Menetelmä toimii
kuitenkin huonosti ympäristössä, jossa ovet, seinät ja muut
esteet vaikuttavat signaalin voimakkuuteen.

Fingerprinting-menetelmässä kerättyyn tietoon sovitetaan
suoraan malli, jonka avulla paikkatieto lasketaan
signaalidatasta.

\subsection{Paikannus sisätiloissa}

Suurissa kerrostaloissa suoritetuista GSM-mittauksista
\texttt{fingerprinting}-menetelmällä saatujen paikkatietojen
mediaanitarkkuus vaihteli tutkimuksessa 2.48 m ja 5.44 m
välillä. WiFi-mittauksista samalla menetelmällä laskettujen
paikkatietojen mediaanitarkkuus oli keskimäärin 23 \%
parempi. Mittausten perusteella onnistuttiin myös
luokittelemaan suurella tarkkuudella rakennuksen kerros,
jossa käyttäjä on.

Sisätilamittausten perusteella yritettiin lisäksi tunnistaa,
missä huoneessa käyttäjä on. Tässä saavutettiin 70 \%
luokittelutarkkuus, joskin suurin osa väärin luokitelluista
huoneista oli heti oikean huoneen vieressä.

\subsection{Paikannus ulkotiloissa}

\texttt{Fingerprinting}-menetelmällä saatiin ulkotiloissa
GSM-paikannuksen mediaanivirheeksi 75 m, mikä artikkelin
mukaan vastaa tyypillistä WiFi-paikannuksen tarkkuutta ja
riittää paikkatietoa ulkotiloissa hyödyntävien sovellusten
tarpeisiin.

\subsection{Paikan tunnistus}

Tutkimuksessa laadittiin menetelmä, jonka avulla
tunnistetaan käyttäjälle merkityksellisiä paikkoja.
Tutkimuksessa näiden katsottiin olevan sellaisia kohteita,
joissa käyttäjä on kerätyn signaalidatan perusteella pitkään
paikallaan.  Kuukauden mittaisen testijakson aikana kerätyn
datan perusteella saavutettiin suuri vastaavuus menetelmän
tunnistamien ja etukäteen merkitykselliksi nimettyjen
paikkojen välillä.

\section{Yhteenveto}

Artikkelin mukaan GSM-verkkoon perustuva paikannus on
varteenotettava ratkaisu paikkatietoa hyväksikäyttävien
sovellusten tarpeisiin sisä- ja ulkotiloissa.
GSM-paikannuksen avulla voidaan myös suurella tarkkuudella
tunnistaa käyttäjälle merkityksellisiä paikkoja.
Menetelmällä saavutetut tulokset ovat tarkkuudeltaan
vastaavia WiFi-paikannuksen kanssa.

Artikkelin mukaan tutkimuksessa saavutetut tulokset ovat
laajennettavissa kaikenlaisiin matkapuhelinverkkoihin, jotka
perustuvat kiinteiden tukiasemien vakioteholla lähet\-tämän
signaalin voimakkuuden havainnointiin.

Menetelmän haittapuolena GPS-verkkoihin verrattuna on
opetusvaiheen vaatima työläs signaali- ja
paikkatietoaineiston kerääminen.

\bibliographystyle{plainnat}
\bibliography{kandi}
\end{document}
