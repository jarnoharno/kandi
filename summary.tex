\documentclass[a4paper]{scrartcl}
\usepackage[finnish]{babel}
\usepackage[utf8]{inputenc}
\usepackage{verbatim}
\usepackage[numbers]{natbib}
\usepackage{hyperref}
\subject{Artikkelireferaatti}
\author{Jarno Leppänen}
\title{Matkapuhelinverkkoon perustuva paikannus 
paikkatietoa käyttävien sovellusten tarpeisiin}
\date{31.1.2014}
\begin{document}
\maketitle

\section{Johdanto}

Varshavsky et al.\cite{Varshavsky06aregsm} esittävät matkapuhelinverkkoihin
perustuvan paikannuksen olevan varteenotettava ratkaisu paikkatietoa
hyödyntävien sovellusten tarpeisiin. Artikkelin mukaan sovellukset hyödyntävät paikkatietoa tyypillisesti kolmella
eri tavalla: sisätiloissa tapahtuvaan paikannukseen, ulkotiloissa tapahtuvaan
paikannukseen sekä merkityksellisten paikkojen tunnistamiseen. Nämä tarpeet 
ovat tyypillisesti erilaisia: sisätiloissa
korostuu paikannuksen tarkkuus ja ulkotiloissa paikannuksen kattavuus.

\section{Taustaa}

Yleisin paikannustekniikka on satelliittipaikannukseen perustuva GPS, joka
on lyömätön avoimissa ulkotiloissa, mutta toimii huonosti sisä- ja kaupunkitiloissa.

WiFi-paikannus on WiFi-verkkojen yleisyyden vuoksi houkutteleva vaihtoehto
paikannuksen tarpeisiin. Artikkelissa esitetään useitä hankkeita, joissa
WiFi-paikannuksen tarkkuudesta on saatu lupaavia tuloksia. WiFi on kuitenkin
usein virrankulutuksen vuoksi kytketty pois päältä, mikä heikentää sen hyödyntämistä
jatkuvaan paikannukseen.

Artikkelin mukaan matkapuhelintekniikalla on useita hyviä puolia paikannuksen
näkökulmasta. Matkapuhelimet ovat aina mukana, jatkuvasti kiinni verkossa ja ovat vakiintunutta tekniikkaa. Matkapuhelinverkot toimivat tarkkaan säädellyillä taajuusalueilla, joten
ne kärsivät WiFi-verkkoa vähemmän erilaisista häiriölähteistä. Matkapuhelinverkoilla on lisäksi WiFi-verkkoja laajempi kattavuus ja verkkoympäristö muuttuu suurten rakentamiskustannusten vuoksi hitaasti.
	  
\section{GSM}
GSM on laajalle levinnyt matkapuhelinverkkostandardi: GSM-verkkoja on artikkelin kirjoittamisen aikaan ollut 210 maassa 676 operaattorin ylläpitämänä.

GSM-verkko jakautuu maantieteellisesti matkapuhelintukiasemien virittämiin soluihin. Kullakin solulla on tietty
kontrollikanava, jolla solut lähettävät vakioteholla signaalia muun muassa naapurisolujen asetuksista. Eri solujen lähettämien signaalien avulla matkapuhelinverkon käyttäjät voivat
vertailla solujen kuuluvuutta ja vaihtaa tarvittaessa solua.

Artikkelissa kehitettiin kahdelle matkapuhelinverkossa toimivalle laitteelle
ohjelmisto, jonka avulla laitteiden soluilta vastaanottamaan signaalidataan päästän
käsiksi.

\section{Paikannus}

Artikkelissa käytettiin fingerprinting- ja centroid-menetelmää paikkatiedon laskemiseen GSM- ja WiFi-verkoissa. Molemmat menetelmät vaativat opetusvaiheen,
jossa tarkkaan paikkatietoon yhdistettyn signaalidatan avulla laaditaan malli, jolla paikkatieto voidaan estimoida signaalidatan perusteella. Paikkatieto saadaan GPS-signaalista.

Centroid-menetelmässä kerätyn tiedon perusteella lasketaan tukiasemien sijainnit,
jolloin uuden signaalitiedon perusteella voidaan estimoida matkapuhelinverkon
käyttäjän sijainti. Menetelmä toimii kuitenkin huonosti ympäristössä, jossa
ovet, seinät ja muut esteet heikentävät signaalia odottamattomalla tavalla.

Fingerprinting-menetelmässä kerättyyn tietoon sovitetaan suoraan malli, jonka
avulla paikkatieto lasketaan signaalidatasta.

\subsection{Paikannus sisätiloissa}

Suurissa kerrostaloissa suoritetuista GSM-mittauksista fingerprinting-menetelmällä saatujen paikkatietojen mediaanitarkkuus vaihteli tutkimuksessa 2.48 m ja 5.44 m 
välillä. WiFi-mittauksista avulla samalla menetelmällä lasketut paikkatietojen 
mediaanitarkkuus oli keskimäärin 23\% parempi. Mittausten perusteella onnistuttiin
myös luokittelemaan suurella tarkkuudella kerros, jossa käyttäjä on.

Sisätilamittausten perusteella yritettiin lisäksi tunnistaa, missä huoneessa
käyttäjä on. Tässä saavutettiin 70\% luokittelutarkkuus, joskin suurin osa
väärin luokitelluista huoneista oli heti oikean huoneen vieressä.

\subsection{Paikannus ulkotiloissa}

Fingerprinting-menetelmällä saatiin ulkotiloissa GSM-paikannuksen mediaanivirheeksi 75m, mikä 
artikkelin mukaan vastaa tyypillistä WiFi-paikannuksen tarkkuutta ja riittää
paikkatietoa ulkotiloissa hyödyntävien sovellusten tarpeisiin.

\subsection{Paikan tunnistus}

Tutkimuksessa laadittiin menetelmä, jonka avulla tunnistetaan käyttäjälle
merkityksellisiä paikkoja. Näiden ajateltiin olevan sellaisia kohteita, joissa
käyttäjä on kerätyn signaalidatan perusteella pitkään paikallaan. Tutkimuksessa
saavutettiin kuukauden mittaisen testijakson aikana kerätyn datan perusteella suuri vastaavuus menetelmän tunnistamien ja etukäteen nimettyjen paikkojen välillä.

\section{Yhteenveto}

Artikkelin mukaan tutkimus osoittaa, että GSM-verkkoon perustuva paikannus
on varteenotettava ratkaisu paikkatietoa hyväksikäyttävien sovellusten tarpeisiin
sisä- ja ulkotiloissa. GSM-paikannuksen avulla voidaan myös suurella 
tarkkuudella tunnistaa käyttäjälle merkityksellisiä paikkoja. Menetelmällä saavutetut tulokset ovat tarkkuudeltaan vastaavia WiFi-paikannuksen kanssa.

Artikkelin mukaan tutkimuksessa saavutetut tulokset ovat laajennettavissa
kaikenlaisiin matkapuhelinverkkoihin.

Menetelmän haittapuolena GPS-verkkoihin nähden on opetusvaiheen vaatima työläs signaali- ja paikkatietoaineiston kerääminen.

\bibliographystyle{plainnat}
\bibliography{kandi}
\end{document}
